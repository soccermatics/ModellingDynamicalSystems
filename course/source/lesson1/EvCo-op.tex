
\subsection{\Large Evolution of co-operation}

\label{evocoop}

{\bf Reading:} Britton (2002) sections 4.9-4.10.

Game theory is used both in biology and in economics to predict how
individuals will act when faced with an interaction with another
individual from which both of them may benefit (or lose out) from. In
economics the aim is to find out the rational behaviour, in biology it
is to find the behaviour selected by natural selection.\\

Here we look at a competition between two behavioural
strategies, one we call 'co-operate' the other 'defect'. In general
co-operators try to share resources and defectors try to take
everything for themselves. We construct first a payoff table based on
the interactions of c-operators (C) and defectors (D). We assume that
when
\begin{itemize}
\item C meets C they both get payoff $1$ (reward).
\item D meets D they both get payoff $0$ (punishment).
\item C meets D then D gets payoff $T$ (temptation), and C gets payoff
$S$ (sucker).
\end{itemize}
In table form

\begin{tabular}{r|cc}
individual/opponent & C & D \\
\hline
C & 1 & $S$ \\
D & $T$ & 0 
\end{tabular}

gives the payoffs for the individuals playing the game.\\

Under the following assumptions we can write down how a population of
these individuals will evolve as a differential equation:
\begin{itemize}
\item Population is infinite.
\item Reproduction is asexual.
\item Pairwise contests occur between two individuals.
\end{itemize}
Let $x$ be the proportion of the population who co-operate. Let the
{\bf fitness} of an individual be its expected payoff given that there
are $x$ co-operators in the population. We assume that the rate of
reproduction is proportional to own fitness minus average fitness
(Darwin' law of natural selection).
\begin{equation}
\frac{dx}{dt} = x (\mbox{fitness of C - average fitness}) \label{repeq}
\end{equation}
this is known as the {\bf replicator equation}.\\

The fitness of C is
\[
x.1+(1-x)S
\]
and the fitness of D is
\[
xT+(1-x).0 = xT
\]
The average fitness is
\[
x(x+(1-x)S) + (1-x)xT
\]
Thus 
\begin{eqnarray}
\frac{dx}{dt} = f(x) & = & x (x+(1-x)S - x(x+(1-x)(S+T))) \nonumber \\
& = & x (1-x) (x+(1-x)S - Tx)
\label{repeqf}
\end{eqnarray}
is the replicator equation for this game.\\

The steady states of equation \ref{repeqf} are $x_*=0$, $x_*=1$ and the
solution to
\[
x_*+(1-x_*)S = Tx_*
\]
or 
\[
x_* = \frac{S}{S+T-1}
\]
In order for this steady state to lie between 0 and 1 we need either
\begin{itemize}
\item $T>1$ and $S>0$\\
\item or $T<1$ and $S<0$\\
\end{itemize}

We can determine the stability of the three steady states by
differentiating equation \ref{repeqf} with respect to $x$.
\[
f'(x) = x(1-x)(1-S-T) + (1-2x)(x+(1-x)S-Tx)
\]
Evaluating at the steady states we get
\[
f'(0) = S
\]
so the 0 steady state is stable if $S<0$. 
\[
f'(1) = T - 1 
\]
so the 1 steady state is stable if $T<1$. 
\begin{eqnarray*}
f'\left(\frac{S}{S+T-1}\right) & = & -S (1 - \frac{S}{S+T-1}) \\ 
& = & \frac{S(T-1)}{1-S-T} 
\end{eqnarray*}
Thus the co-existence steady state is stable if both $T>1$ and
$S>0$, but unstable if $T<1$ and $S<0$.\\

Below we illustrate how the stability is determined in
the $ST$ plane.\\

\vspace{5cm}

\newpage

We now consider some examples. The Hawk-Dove game is

\begin{tabular}{r|cc}
individual/opponent & Dove & Hawk \\
\hline
Dove & 1 & $S=1/4$ \\
Hawk & $T=5/4$ & 0 
\end{tabular}

Thus $S>0$ and $T>1$, so the hawks and doves coexist at $1/2$. Note
that this is not optimal for the population as a whole. The mean
fitness in this situation is $1/2(1/2+1/2 \times 1/4)+5/4 \times
1/2\times1/2=5/16+5/16=10/16$. If all played Dove the mean fitness would be
$1$. This shows that natural selection acts to maximise individual
fitness and not group fitness.\\

The prisoners dilemma is

\begin{tabular}{r|cc}
individual/opponent & Keep quiet & Blame other \\
\hline
Keep quiet  & 1 & $S=-1/4$ \\
Blame other & $T=5/4$ & 0 
\end{tabular}

Here $S<0$ and $T>1$ and the stbale strategy is to keep quiet. Here
the situation is very bad for the population as a whole. The group
could do best by all co-operating but by acting rationally or
following natural selection all will defect.\\

The stag hunt is

\begin{tabular}{r|cc}
individual/opponent & Group & Self \\
\hline
Group & 1 & $S=-1/4$ \\
Self & $T=1/2$ & 0 
\end{tabular}

thus $S<0$ and $T<1$, so all defect and all co-operate are both
stable. The steady state
\[
\frac{S}{S+T-1} = \frac{1}{3}
\]
thus if $x(0)>1/3$ then all defect, if $x(0)<2/3$ then all
co-operate. Stag hunts are hard to establish but when established are
stable to defection.
