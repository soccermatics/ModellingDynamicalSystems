

\section{Räkneövningar till kapitel 8 i kompendiet}


\subsection{}
Vi undersöker ytligare två spel teori modeller med hjälp av replicator ekvationen.
\begin{eqnarray*}
\dot{x}(t) & = &  x(t) (1-x(t)) (x(t)(1 - S - T) + S) 
\label{repeqf}
\end{eqnarray*}
\begin{itemize}
\item Hawk Dove game:
\center
\begin{tabular}{r|cc}
du/andra & C & D \\
\hline 
C  & 1 & $S=1/2$ \\
D & $T=3/2$ & 0 
\end{tabular}\\[1mm]

\flushleft
\item Stag Hunt game:
\center
\begin{tabular}{r|cc}
du/andra & C & D \\
\hline 
C  & 1 & $S=-1/4$ \\
D & $T=1/2$ & 0 
\end{tabular}\\[1mm]
\end{itemize}
För både modelerna hitta alla (realistiska) jämviktspunkter och använder linjärisering för att bestäm stabilitet.  Skriva en kort tolkning av det som du har kommit fram till om modellerna.

\subsection{}
%%%%%%%%%%%%%%%%%%%%%%%%%%%%%%%%%%%%%%%%%%%%%%%%%%%%%
%  13.1
%%%%%%%%%%%%%%%%%%%%%%%%%%%%%%%%%%%%%%%%%%%%%%%%%%%%%
Givet differential\-ekvationen
$$
\ddot{x} - (0.1 - \frac{10}{3} \dot{x}^2 ) \dot{x} + x + x^2 = 0
$$
Inför tillståndsvariablerna $x_1 = x $ och $x_2 = \dot{x} $, och skriv  systemet på tillståndsform. 
Bestäm de stationära punkterna, linjärisera tillståndsmodellen kring dessa och undersök de stationära punkternas karaktär.
Ta hjälp av kapitel 8.3.1 ''Fasportträtt för linjära system''. 
Vad gäller för det olinjära systemet i närheten av jämviktspunkterna?  Se kapitel 8.4.1 ''Fasporträtt nära jämviktspunkt'' i kompendiet.
Vad kan man säga om beteendet längre bort från jämviktspunkterna?  Se kapitel 8.4.2 ''Fasporträtt långt från jämviktspunkter'' i kompendiet.
Skissa fasplanet.


\subsection{}
En variant av Lotka-Voltera modellen är
\begin{eqnarray*}
\frac{dx}{dt} & = & f(x,y) = bx-xy - ax^2  \\
\frac{dy}{dt} & = & g(x,y) =xy-dy
\end{eqnarray*}
Skriv fasportraiter för modellerna (det finns minst två olika 'typ' av fasportrait),  hitta jämviktspunkter och bestäm stabilitet.

\subsection{}
Betrakta systemet
\[
\dot{x} = \left(
\begin{array}{c}
-x_1^3 + u \\ x_1
\end{array} \right)
\]

Skissa fasplanet för $u = 0$.\\
Ledning: Utnyttja ekvation (7.22) i kompendiet.


\subsection{}
En bils dynamik kan via Newtons 2:a lag beskrivas med
differentialekvationen
\begin{equation*}
m \dot{v}(t)=F_d(t)-bv^2(t)
\end{equation*}
där $v$ är bilens hastighet, $F_d$ är den dragkraft motorn ger
(insignalen) och $bv^2(t)$ är den motkraft som genereras av
luftmotståndet ($b$ är här en proportionalitetskonstant).
\begin{itemize}
\item[a)] Linjärisera systemet runt en godtycklig stationär punkt $\{F_{d,0},v_0\}$

\item[b)] Antag att den konstanta (stationära) insignalen är
  $F_{d,0}$=1. Tag med hjälp av resultatet i a) fram
  överföringsfunktionen mellan avvikelsen från 
  insignalens jämviktsläge och avvikelsen från utsignalens
  (hastighetens) jämviktsläge. Under vilken förutsättning är systemet
  stabilt nära jämviktspunkten? Kan man förvänta sig att detta krav är
  uppfyllt?  
\end{itemize}


\subsection{}
I processlabben stötte ni på dubbeltankprocessen. I princip
kan ett system av 
olinjära differentialekvationer som beskriver vätskehöjderna i tankarna
skrivas som
\begin{align*}
\dot{h_1}(t)&=-\frac{a}{A}\sqrt{2 g h_1(t)}+\frac{K_P}{A}u(t)\\
\dot{h_2}(t)&=\frac{a}{A}\sqrt{2 g h_1(t)}-\frac{a}{A}\sqrt{2 g h_2(t)}
\end{align*}
där $h_1$ är vätskehöjden i övre tanken, $h_2$ är vätskehöjden i undre
tanken, $u(t)$ är spänningen på den lilla motor som pumpar vatten till
övre tanken. Övriga variabler är konstanter, $a/A$ är kvoten mellan
bottenhålets area och tankens tvärsnittsarea (som vi antar vara samma
för båda tankarna), $K_P/A$ [cm/s/V] är förhållandet mellan en motorkonstant
och tankarean, och $g$ är tyngdaccelerationen $981$ [cm/s$^2$]. I ett
lab-experiment har vi erhållit $a/A$=0.015 och $K_P/A$=0.125. Sätter
man in dessa värden i modellen ovan får man:

\begin{align*}
\dot{h_1}(t)&=-0.06644\sqrt{h_1(t)}+0.125u(t)\\
\dot{h_2}(t)&=0.0664\sqrt{h_1(t)}-0.0664\sqrt{h_2(t)}
\end{align*} 

\begin{itemize}
\item[a)] Antag nu att de tillstånd vi väljer är vätskehöjderna i
  respektive tank. Tag fram en linjär modell som beskriver systemet
  (eg avvikelsen från jmv.punkten $x-x_0$) i
  ett område runt den stationära punkt som fås
  då insignalen u(t) är konstant 2 V, dvs $u=u_0=2$.

\item[b)] Antag nu att den utsignal vi vill studera är vätskenivåns
  avvikelse från jämviktsläget i
  den undre tanken ($y-y_0$). Tag fram överföringsfunktionen från
  insignalens avvikelse från jämviktspunkten ($u-u_0$) till
  $y-y_0$. Är systemet stabilt?

\end{itemize}

\subsection{}
En olinjär tillståndsmodell ges av 
\begin{align*}
 \begin{bmatrix} \dot{x}_1(t)\\\dot{x}_2(t)\end{bmatrix}&=\begin{bmatrix}
 -\sqrt{x_1(t)}+x_2(t)\\x_1(t)u(t)-x_2(t)\end{bmatrix} \\
y(t)=x_1^2(t)
\end{align*}
Då $u(t)=$konstant$=0.5$ har systemet en jämviktspunkt i $x_{1,0}=4$ och
$x_{2,0}=2$. Verifiera detta, och tag sedan fram en linjär modell som
beskriver systemet i ett område nära jämviktspunkten. Kan man utifrån
denna dra
några slutsatser om hur lösningarna kommer att se ut nära
jämviktspunkten, d v s vilken typ av jämviktspunkt det olinjära
systemet har? 




