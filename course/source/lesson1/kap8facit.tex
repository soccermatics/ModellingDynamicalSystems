\section{Olinjära system}
\subsection{} kommer snart

\subsection{}
%%%%%%%%%%%%%%%%%%%%%%%%%%%%%%%%%%%%%%%%%%%%%%%%%%%
%  13.1
%%%%%%%%%%%%%%%%%%%%%%%%%%%%%%%%%%%%%%%%%%%%%%%%%%%
$$
\ddot{x} - (0.1-10\dot{x}^2/3)\dot{x} + x + x^2=0
$$
Inför tillståndsvariablerna
$x_1 = x $ och $x_2 = \dot{x} $, samt skriv på
tillståndsform

\[
\begin{array}{ccl}
\dot{x}_1 &=& x_2 = f_1(x_1, x_2) \\
\dot{x}_2 &=& -x_1(1+x_1) + x_2(0.1-10x_2^2/3) = f_2(x_1, x_2)
\end{array}
\]
Bestäm de stationära (singulära) punkterna och deras karaktär, samt
skissera fasplanet. (I lösningen nedan betecknas stationära punkter med $x^o$ istället för som i kompendiet
$x_o$, detta för att slippa "dubbelindex")
\begin{enumerate}
\item {\bf Stationära punkter} \\
$f(x^\circ ) = 0\ \ \Rightarrow\ \ 
x{_2^\circ} = 0$ samt $x{_1^\circ} (1+x{^\circ _1}) = 0$.
$$
{\rm SP \  I}\ \left\{ \begin{array}{l} x{^\circ_1} = 0 \\ x{^\circ_2}
= 0 \end {array} \right. ,\qquad \qquad {\rm SP \ II}\  \left\{
\begin{array}{l} x{^\circ_1} = -1 \\ x{^\circ_2} = 0 \end{array}
\right.   
$$
\item {\bf Linjärisera kring de stationära punkterna} \\
Taylors formel:
$$
f(x) = f(x^\circ ) + \left. \frac{df}{dx}(x) \right|_{x=x^\circ}
(x-x^\circ) + o|x-x^\circ| = \left. \frac{df}{dx}(x) 
\right|_{x=x^\circ} (x-x^\circ) + o|x-x^\circ|
$$
eftersom $f(x^\circ )=0$.
Matrisen $\frac{df}{dx}(x)$ är funktionen $f$:s
Jacobian. Den har $ij$-elementet
$\frac{\partial f_i}{\partial x_j}(x)$
$$
\renewcommand{\arraystretch}{2}
\begin{array}{ll}
\frac{\partial f_1}{\partial x_1} = 0\ \ \ &
\frac{\partial f_1}{\partial x_2} = 1 \\
\frac{\partial f_2}{\partial x_1} = -1 -2 x_1 \ \ \ &
\frac{\partial f_2}{\partial x_2} = 0.1 - 10 x_2^2
\end{array}
\renewcommand{\arraystretch}{1}
$$
Gör variabelbytet $z = x - x^\circ$ i de olika
stationära punkterna.
\item {\bf SP I}\\
Linjär approximation $\dot{z} = A_{SP1}z$, med
$$
A_{SP1} = \begin{pmatrix} 0 & 1 \\ -1 & 0.1 \end{pmatrix}
$$
Matrisen $A_{SP1}$:s egenvärden ges av
\[0 = \det (\lambda I - A_{SP1}) = \lambda(\lambda-0.1) + 1,\]
dvs
\[ \lambda= 0.05 \pm \sqrt{0.05^2 - 1} \]

Den linjära approximationen har således ett instabilt
fokus i SP1 dvs (0,0). För instabila fokus gäller att
den olinjära differentialekvationen har samma typ av
singularitet som den linjära approximationen (se kursmaterialet). 
Observera att den linjära approximationen endast gäller 
\emph{nära} den stationära punkten.

\begin{figure}
\centering
\includegraphics[width=0.44\linewidth]{figures/u15_1a.eps}
\includegraphics[width=0.44\linewidth]{figures/u15_1b.eps}
\caption{Fasporträtt för SPI. Till vänster ses den länjära approximationen, och till höger visas fasporträttet beräknat numeriskt, ickelinjärt.}
\end{figure}

\item {\bf SP II} \\
Linjär approximation $\dot{z} = A_{SP2}z$, med
$$
A_{SP2} = \begin{pmatrix} 0 & 1 \\ 1 & 0.1 \end{pmatrix}
$$
Egenvärden till $A_{SP2}$,
\begin{eqnarray*}
0 &=& {\rm det}(\lambda I -A_{SP2}) = \lambda (\lambda -0.1) - 1 \\
\lambda &=& 0.05 \pm \sqrt{0.05^2 + 1}, \quad \lambda_1 \approx -0.95, \quad
\lambda_2 \approx 1.05 
\end{eqnarray*}
Den linjäriserade ekvationen har tydligen en sadelpunkt
i (-1,0). Detta gäller även för den olinjära
ekvationen.
Den stabila egenvektorn är (fås t ex från kommandot "eig" i matlab)  $(1,-0.95)$, och den instabila
är $(1,1.05)$.

\begin{figure}[ht]
\centering
\includegraphics[width=0.8\linewidth]{figures/13.1.II.eps}
\caption{Fasporträtt för SPII}
\end{figure}

\item{\bf Långt från stationära punkter} \\
Hur ser banorna ut på långt avstånd från origo? 
Bilda derivatan,
$$
\frac{d x_2}{d x_1} =
\frac{\dot{x}_2}{\dot{x}_1} = \frac{-x_1(1+x_1) + 
x_2(0.1 -10x_2^2/3)}{x_2} 
$$
När $x_1$ är begränsad och
$x_2 \rightarrow \pm \infty $, så gäller tydligen
att $\dot{x}_2/\dot{x}_1 \rightarrow -\infty$. Alltså blir
banorna lodräta när $|x_2|$ växer (och även
då $x_2 \rightarrow 0$).
\end{enumerate}

\begin{center} \includegraphics[width=0.8\linewidth]{figures/13.1.tot.eps} \end{center}

\subsection{} kommer snart


\subsection{}
 Stationära (singulära) punkter ges av $x_1 = 0$, dvs hela
  $x_2$-axeln, då $u=0$. Banorna ges av 
  \begin{displaymath}
    \frac{d x_2}{d x_1} = -\frac{1}{x_1^2}
    \qquad \Leftrightarrow \qquad
    x_2 = \frac{1}{x_1} + C.
  \end{displaymath}
\begin{center}\includegraphics[width=0.8\linewidth]{figures/13.7.1.eps}\end{center}



\subsection{}
\begin{itemize}
\item[a)] 
\begin{equation}
\Delta \dot{v}(t)=\frac{-2b}{m}v_0\Delta v(t)+\frac{1}{m}\Delta F_d(t)
\end{equation}
där $\Delta v(t)=v(t)-v_0$ och $\Delta F_d(t)=F_d(t)-F_{d,0}$

\item[b)]
\begin{equation}
\Delta V(s)=\frac{1/m}{s+2\sqrt{b}/m}\Delta F_d(s)
\end{equation}
Asymptotiskt stabil då $b>0$ vilket i detta fall alltid bör gälla.
\end{itemize}


\subsection{}
\begin{itemize}
\item[a)]
Sätt $h=\begin{bmatrix} h_1(t) & h_2(t) \end{bmatrix}^T$. Jämviktshöjderna motsvarande $u_0=2$ blir
$h_{1,0}=14.1$ cm  och $h_{2,0}=14.1$ cm. Med $\Delta h(t)=h(t)-h_0$
kan det linjariserade systemet skrivas

\begin{align*}
\Delta \dot{h}(t)=\begin{bmatrix} -\frac{K}{2\sqrt{h_{1,0}}}& 0\\
\frac{K}{2\sqrt{h_{1,0}}}&-\frac{K}{2\sqrt{h_{2,0}}}\end{bmatrix} \Delta h(t)+
\begin{bmatrix} K_m\\0\end{bmatrix} \Delta u(t)
\end{align*}
där $K=0.06644$ och $K_m=0.125$. Med alla värden insatta blir systemet
\begin{align*}
\Delta \dot{h}(t)=\begin{bmatrix} -0.0088& 0\\
0.0088&-0.0088\end{bmatrix} \Delta h(t)+
\begin{bmatrix} 0.125\\0\end{bmatrix} \Delta u(t)
\end{align*} 

\item[b)]
$\Delta y(t)=\Delta h_2(t)=\begin{bmatrix} 0&1\end{bmatrix}\Delta h(t)$.
Sambandet mellan avvikelse i in och utsignal fås av $\Delta
Y(s)=C(sI-A)^{-1}B \Delta U(s)$. Man erhåller med $A$, $B$ och $C$
enligt ovan att 
\begin{equation}
\Delta Y(s)=\frac{0.0011}{s^2+0.0176 s+7.744\dot10^{-5}}\Delta U(s)
\end{equation}
Dubbelpol i $\lambda_{1,2}=-0.0088$ och jämviktspunkten därmed stabil.
(Polerna för det linjariserade systemet hade också kunnat erhållas direkt från
$A$-matrisens egenvärden).

\end{itemize}

\subsection{}
$\dot{x}(t)=0$ och insättning verifierar jämviktspunkterna.
Den linjära modellen kan skrivas
\begin{align*} 
\Delta \dot{x}(t)&=\begin{bmatrix} -\frac{1}{2 \sqrt{x_{1,0}}}& 1\\u_0&-1\end{bmatrix}
\Delta x(t)+\begin{bmatrix} 0 \\x_{1,0}\end{bmatrix} \Delta u(t)\\
\Delta y(t)&=\begin{bmatrix} 2 x_{1,0}\;\; 0\end{bmatrix} \Delta x(t)
\end{align*}
Med värden insatta fås att systemmatrisen $A$ är
\begin{equation*}
A=\begin{bmatrix} -0.25 &1\\0.5&-1\end{bmatrix}
\end{equation*}
Egenvärdena till denna (polerna till det linjäriserade systemet) ges
av polynomet $(s+0.25)(s+1)-0.5=0$,  d v s
$s=0.175$ och $s=-1.425$. Punkten är alltså en sadelpunkt, och 
lösningarna till det olinjära systemet beter sig därmed på liknande
sätt runt jämviktspunkten. 

 
