\hypertarget{uppgifter-puxe5-svenska}{%
\section{Uppgifter (på svenska)}\label{uppgifter-puxe5-svenska}}

\begin{enumerate}
\item
  Givet differentialekvationen

  \[\ddot{x} - (0.1 - \frac{10}{3} \dot{x}^2 ) \dot{x} + x + x^2 = 0\]

  Inför tillståndsvariablerna \(x_1 = x\) och \(x_2 = \dot{x}\), och
  skriv systemet på tillståndsform. Bestäm de stationära punkterna,
  linjärisera tillståndsmodellen kring dessa och undersök de stationära
  punkternas karaktär.
\item
  En variant av Lotka-Voltera modellen är

  \[\begin{aligned}
  \begin{aligned}
     \frac{dx}{dt} & = f(x,y) = bx-xy - ax^2  \\
     \frac{dy}{dt} & = g(x,y) =xy-dy
  \end{aligned}
  \end{aligned}\]

  Skriv fasportraiter för modellerna (det finns minst två olika 'typ' av
  fasportrait), hitta jämviktspunkter och bestäm stabilitet.

  \begin{enumerate}
  \tightlist
  \item
    Undersök SIRS modellen:
  \end{enumerate}
\end{enumerate}

\begin{quote}
\[\begin{aligned}
\begin{aligned}
\frac{dS}{dt} & = & - \beta S I + \alpha R\\
\frac{dI}{dt} & = & \beta S I  -  \gamma I \\
\frac{dR}{dt} & = & \gamma I - \alpha R
\end{aligned}
\end{aligned}\]\[Rita en fasporträtt för modellen för fallet :math:`\beta>\gamma`.
Bestäm ’nullclines’ för modellen. Använd :math:`R=1-S-I` för att
substituera bort :math:`R` i modellen ovan.
Hitta jämviktspunkten för systemet. Anta att :math:`\beta=1/5` och
:math:`\gamma =1/10`. Visa att (oavsett värdet för :math:`\alpha`)
jämviktspunkten är asymptotiskt stabil.\]\[**Extra utmaning :** Ange villkor för att jämviktspunkten är en stabil spiral.\]

\begin{enumerate}
\tightlist
\item
  Betrakta systemet
\end{enumerate}

\begin{quote}
\[\begin{aligned}
\dot{x} = \left(
 \begin{array}{c}
 -x_1^3 + u \\ x_1
 \end{array} \right)
\end{aligned}\]

Skissa fasplanet för \(u = 0\).
\end{quote}
\end{quote}

\begin{enumerate}
\item
  I processlabben stötte vi på dubbeltankprocessen. I princip kan ett
  system av olinjära differentialekvationer som beskriver vätskehöjderna
  i tankarna skrivas som

  \[\begin{aligned}
  \begin{aligned}
  \dot{h_1}(t)&=-\frac{a}{A}\sqrt{2 g h_1(t)}+\frac{K_P}{A}u(t)\\
  \dot{h_2}(t)&=\frac{a}{A}\sqrt{2 g h_1(t)}-\frac{a}{A}\sqrt{2 g h_2(t)}
  \end{aligned}
  \end{aligned}\]

  där \(h_1\) är vätskehöjden i övre tanken, \(h_2\) är vätskehöjden i
  undre tanken, \(u(t)\) är spänningen på den lilla motor som pumpar
  vatten till övre tanken. Övriga variabler är konstanter, \(a/A\) är
  kvoten mellan bottenhålets area och tankens tvärsnittsarea (som vi
  antar vara samma för båda tankarna), \(K_P/A\) {[}cm/s/V{]} är
  förhållandet mellan en motorkonstant och tankarean, och \(g\) är
  tyngdaccelerationen \(981\) {[}cm/s\(^2\){]}. I ett lab-experiment har
  vi erhållit \(a/A\)=0.015 och \(K_P/A\)=0.125. Sätter man in dessa
  värden i modellen ovan får man:

  \[\begin{aligned}
  \begin{aligned}
  \dot{h_1}(t)&=-0.06644\sqrt{h_1(t)}+0.125u(t)\\
  \dot{h_2}(t)&=0.0664\sqrt{h_1(t)}-0.0664\sqrt{h_2(t)}
  \end{aligned}
  \end{aligned}\]

  \begin{description}
  \item[* Antag nu att de tillstånd vi väljer är vätskehöjderna i
  respektive]
  tank. Tag fram en linjär modell som beskriver systemet (eg avvikelsen
  från jmv.punkten \(x-x_0\)) i ett område runt den stationära punkt som
  fås då insignalen u(t) är konstant 2 V, dvs \(u=u_0=2\).
  \item[* Antag nu att den utsignal vi vill studera är vätskenivåns
  avvikelse]
  från jämviktsläget i den undre tanken (\(y-y_0\)). Tag fram
  överföringsfunktionen från insignalens avvikelse från jämviktspunkten
  (\(u-u_0\)) till \(y-y_0\). Är systemet stabilt?
  \end{description}
\end{enumerate}
