
\section{Modellering av epidemier} 

I denna uppgift kommer vi att jobba med olika varianter av SIR-modellen.
 
\textbf{Uppgift 4.1:}\\  %\textbf{Uppgift 3.6:}\\
Simulera (med hjälp av \mcode{simSIR.m} och \mcode{SIR.m} ) SIR-modellen:
$$\begin{aligned}\frac{dS}{dt} & = - \beta S I \\
\frac{dI}{dt} & = \beta S I -  \gamma I \\
\frac{dR}{dt} & =  \gamma I 
\end{aligned}$$
Anta att $\gamma=1/7$, $S(0)=999/1000$ och $I(0)=1/1000$. Testa att simulera med olika smittsamhet $\beta=1/3, 1/6, 1/10$ och ange slutvärdet för $R$. För vilket värde på $\beta$ bli det ingen epidemi (dvs för vilket värde på $\beta$ minskar alltid $I$)? Varför?

\ansbox{100}


\textbf{Uppgift 4.2:}\\
Skriv om koden i Matlab för att simulera SEIR modellen (som diskuteras i föreläsning 10):
\[
\begin{aligned}
\frac{dS}{dt} & = - \beta S I \\
\frac{dE}{dt} & = \beta S I - \delta E\\
\frac{dI}{dt} & = \delta E -  \gamma I \\
\frac{dR}{dt} & =  \gamma I 
\end{aligned}
\]
Tillstånd $E$ kallas för \textit{exponerad}: personen har fått viruset men smittar under en kort period ännu inte andra. Anta att $\gamma=1/7$ och $\beta=1/5$.  Anta att tidsenheten är dagar, dvs $\gamma=1/7$ betyder att det i snitt tar 7 dagar att tillfriskna.  Skissa (eller klistra in från Matlab) en graf som plottar $I(t)$ som en funktion av tiden $t$ när man är exponerad ($E$) i snitt under 1, 5 respektive 9 dagar innan man blir infekterad ($I$). Anta att $S(0)=999/1000$, $E(0)=0$ och $I(0)=1/1000$. Har $\delta$ en stor eller liten effekt på slutvärdet för $R$? Förklara ditt svar.\\

\ansbox{200}

Hälsomyndigheten bestämmer att införa restriktioner först när 1\% av befolkning är infekterad. Med restriktioner är $\beta=1/15$ men utan restriktioner är $\beta=1/5$. Undersök konsekvenser av detta beslut för olika värden på $\delta$, dvs simulera spridning först under antagandet att $\beta=1/5$ och när $I(0)=0.01$ ändra smittsamheten till $\beta=1/15$. Skissa $R(t)$ för olika $\delta$-värden och beskriv hur $\delta$ påverkar utfallet.\\

\textbf{Tips:} Använd
\begin{lstlisting}
ind = find(X(:,3)>=0.01);
onepercent=ind(1);
\end{lstlisting}
för att hitta tidpunkten, $t$, då $I(t)=0.01$.
 
\ansbox{150}

\textbf{Uppgift 4.3 \redb{T}:}\\
Undersök nu SIRS modellen:
\[
\begin{aligned}
\frac{dS}{dt} & = - \beta S I + \alpha R\\
\frac{dI}{dt} & = \beta S I  -  \gamma I \\
\frac{dR}{dt} & =  \gamma I - \alpha R
\end{aligned}
\]
Rita en fasporträtt för modellen för fallet $\beta>\gamma$. Bestäm 'nullclines' för modellen. Använd $R=1-S-I$ för att substituera bort $R$ i modellen ovan.

\ansbox{150}

Hitta jämviktspunkten för systemet. Anta att $\beta=1/5$ och $\gamma =1/10$. Visa att (oavsett värdet för $\alpha$) jämviktspunkten är asymptotiskt stabil. 

\ansbox{300}

Skissa (eller klistra in från Matlab) en graf som plottar $I(t)$ som en funktion av tiden $t$ för två olika fall - när $\alpha=1/100$ respektive $\alpha=1/4$. I vilket fall finns det en 'andra våg'?

\ansbox{200}
 

{ \bf Extra utmaning \redb{T}\redb{T}\redb{T}:} (bara om du tycker att det intressant) Ange villkor för att jämviktspunkten är en stabil spiral. 

\ansbox{300}

